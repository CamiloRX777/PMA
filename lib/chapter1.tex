Integral indefinida \cite*{analisismatematico}: Si una función $f(x)$ 
está definida en el intervalo (a, b) y es continua y $F'(x) = f(x)$ para 
$a < x < b$, entonces 
\begin{align*}
    \int f(x) dx = F(x) + C, a < x < b
\end{align*}

Aplicando la tabla de las integrales elementales,
hallar las siguientes integrales.

\begin{enumerate}
    \item $ \displaystyle \int \frac{x+1}{\sqrt{x}} dx $
    \item $ \displaystyle \int  x^5(x^4 - 2)^2dx$
    \item $ \displaystyle \int \frac{3x^3+5x^2+2x+3}{3x+2}dx$
    \item $ \displaystyle \int \pi^{ex}\cdot e^{\pi x}dx$
    \item $ \displaystyle \int \frac{(\sqrt{3x}-\sqrt[3]{3x})^2}{x} dx $
    \item $ \displaystyle \int (1 + \sin{x} + \cos{x}) dx $
    \item $ \displaystyle \int \frac{dx}{1+3x^2} $
    \item $ \displaystyle \int \frac{dx}{\sqrt{2-5x}} $
    \item $ \displaystyle \int \frac{dx}{(2x-2)^2} $
    \item $ \displaystyle \int \frac{xdx}{3-2x^2} $
    \item $ \displaystyle \int  x\cdot e^{-x^2}dx$
    \item $ \displaystyle \int  \frac{1}{\cos{x}}-\dfrac{\cos{x}}{1+\sin{x}}dx$
    \item $ \displaystyle \int  \sin(2x)\cos(3x)dx$
    \item $ \displaystyle \int  \cos(2x)\cos(3x)dx$
    \item $ \displaystyle \int  \sin(2x)\sin(3x)dx$
\end{enumerate}
